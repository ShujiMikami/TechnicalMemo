\documentclass{jsarticle}
\usepackage[dvipdfmx]{hyperref}
\usepackage{pxjahyper}
\hypersetup{colorlinks=true,linkcolor=blue}

\title{MacPro1,1(2006) El Capitan インストール}

\begin{document}
\maketitle

\section{背景}
実家で眠っていたMacPro2006は, Xeonの2.66GHz CPUをデュアルで搭載し, HDDベイは4つ, メモリベイは8あり, HW的性能, 拡張性ともに, 実使用中のMacminiや, MacBookAirと比較しても全く見劣りしないか勝っている\cite{MacPro_Doc}. 
また, メモリも今では古い, DDR2-FBという規格のものであり, 大容量のメモリが格安で手に入る. 今回は, Amazonで, 32GBを約1万円で購入した. 

一方で, 搭載されていたOSは, OS10.4(Tiger)であり, 日常的に利用しているソフトウェアがほとんどインストールできない. 
また, MacPro2006はそのHW特性上, 無条件でアップグレードできるのは, 10.7(Lion)までである\cite{EFI_Issue_Doc}が, 
OSXは, 10.8(Mountain Lion)以前のものはアップグレードが無償でなく, 購入しなければならない. 

普通であればここでもう, MacPro2006は寿命を終える, ということになるが, ネット上の情報によると, 10.11(El Capitan)までは頑張ればアップグレード可能であると言われている\cite{Summary_Doc}. 

なお, El Capitanの次以降の, Sierra, High Sierraについては, SSE4.1 という仕組みが必要で, そこまでいくと, もはやMacPro2006は, HWレベルでサポート外になり, 不可能らしい\cite{Summary_Doc}

\section{概要}
今回, 先行情報に基づき, El Capitanまでのアップグレードを行うことに成功したが, いくつか引っかかる部分もあったので, 備忘録も兼ねて残しておく. 

\section{結論}
El Capitanが普通にインストール可能なMacがあれば, そのPCに, MacProのHDDを接続してEl Capitanをインストールする方法でうまくいく. 

\section{制約}
所詮は裏技なので, El Capitanのインストールができたとはいえ, 制約がある. 
\begin{itemize}
  \item OSのセキュリティアップデートをしてはいけない\cite{Security_Update_Issue_Doc}
  \item Sierraへのアップグレードはできない\cite{Summary_Doc}
\end{itemize}
つまりは, インストールした段階よりも新しくはできない, ということである. 
最新の環境を使いたいなら, それに対応したマシンを買うしかない. 

\section{残存不明点}
なぜ1回目は失敗したのか. 

boot.efiを再ロックかけてもちゃんと動くのか. 

\section{MacPro2006のOS制限}
MacPro2006は, 10.7(Lion)までしかインストールできない\cite{EFI_Issue_Doc}. これは, かなり根っこの部分のアーキテクチャが, 変わっていることによる. 
Macは, PowerPCから, Intelアーキテクチャに移行した際, ファームウェアとOSの間に, EFI(Extensible Firmware Interface)を挟む構造となった(Snow Leopardかららしい). 
そして, OSのアーキテクチャが, 10.8以降, 完全に64bit化した\cite{EFI_Issue_Doc}ことにより, EFIも64bitのものになっていなければならず, EFIが32bitのマシンは, Lion以降のアップグレードができなくなったというわけである. 

\section{MacPro2006のアップグレード可能性}
一方で, MacのHWは, Intelアーキテクチャに移行した初期段階から, HWとしては64bitに対応したものとなっており\cite{MacPro_Doc}, 原理的には64bit動作が可能である. 
つまり, EFIだけが, 32bitの皮をかぶっているに過ぎない. そこに目をつけて, 32bitEFIを64bitEFIとしてふるまうようにラッピング構造を作った人がおり\cite{Pike}, 
そこから派生した様々な手法が確立され, ネット上に拡散することとなった. 

\section{基本コンセプト}
OSXのディレクトリ構造の中に, OSを起動する際のEFIが, boot.efiというファイル名で格納されている. 
この, boot.efiを, Pike氏が64bitラッピングした32bitEFIに置き換えてやれば, MacPro2006でも64bitOSが動作する, というものである. 
ただし, OSXは, インストーラの入手, および, インストーラの実行それぞれの時点で, HWがOSに対応しているものかどうかをチェックする機構を持ち, 単純にはダウンロードすらさせてもらえない. 
この関門を突破するというのも課題となっており, 様々な方法が提案されている. 

\section{前提条件}
グラフィックカードが, 64bitEFI, 32bitEFI両対応している必要がある. 
デフォルトでついている, GeForce7300は, 32bitにしか対応していないため, 使用できない\cite{Graphics_Board_Doc}. 
今回は, MacPro2008の購入オプションとして存在した, ATI Radeon 5770 for Macを用意した. ATI Radeon 5770は, いくつか種類がある\cite{Alternative_GPU}が, for Macであれば, 一切の追加作業なくあっさり動く. メルカリで8000円で入手. 

\section{本備忘録で採用した方法}
様々な手法の中で, やっていることが, 最も直感的に理解しやすい方法を示す. 

\subsection{作業概要}
El Capitanを普通にインストールすることが可能なマシンで, MacProのHDDにEl Capitanをインストールし, HDD内部のboot.efiを置き換える. 

\subsection{必要機材}
\subsubsection{El Capitanがインストール可能な別のMac}
El Capitanがインストール可能なMacが別に存在している必要がある(仮想環境でなんとかしている人もいた). 
今回は, MacBookAir Mid2011を使用した. 
ネット上の記事によっては, El Capitanがインストールされていないとダメ, という情報もあるが, それはガセ. 
今回の作業においては, MacBookAirには, High Sierraが入っていた. 

\subsubsection{SATA-USB変換機}
MacProの起動ドライブを取り出し, 上記Macに直接つなぐので, SATA-USBトランシーバーが必要になる. 
今回は, サンワサプライの, USB-CVIDE3を用意した. ヨドバシで2400円. 

\subsubsection{USBメモリ}
El CapitanをHDDにインストールするためのメディアが必要. 
既存のHDDのパーティションを切ったり, そもそも別MacがEl Capitanであれば必要ないかもしれないが, 大して高いものでもないので用意したほうが無難. 
コンビニで売っている, 16GBのものを用意した. 
最低でも8GBはほしい(インストーラファイルが約6GB)

\subsection{手順}
\subsubsection{El Capitanのダウンロード}\label{sequence_download}
別Macで作業する. 
Appleの公式サイトで, El Capitanをダウンロードする. 
\url{https://support.apple.com/ja-jp/HT206886}

ネットの情報ソースによっては, 過去にダウンロードしたことがあれば, AppleStoreの購入済みから, ダウンロードするのもよい, となっているが, 
今回, それでダウンロードしたEl Capitanのインストーラで, 先に進めると, インストールの際に, 「インストールできるパッケージがありません」というエラーが出て, インストールができなかった. 公式HPからのダウンロードをしたほうがよい. 

\subsubsection{El Capitanのブータブルインストールメディア作成}
別Macで作業する. 
\ref{sequence_download}でダウンロードされたインストーラ(拡張子dmg)を実行すると, パッケージが表示されるので, それをさらに実行する. 
すると, El Capitanの「インストーラがインストール」される. Applicationフォルダを覗くと, El Capitainインストールという項目が追加されている. 

しかし, このインストーラは, El Capitanよりも新しい環境では実行することができず, 「インストールするにはOSが新しすぎます」というエラーメッセージが出る. 
そこで, このインストーラから, PC起動時に実行可能なブータブルメディアを作成する. その方法は, 公式にあり, 
\url{https://support.apple.com/ja-jp/HT201372}
に記述があるように, createinstallmediaというコマンドを実行することでできる.
Volumes/MyVolumeを, USBメモリのパスに置き換える. 
ターミナルに, Doneと表示されるまでしばらく待つ. 
ここで失敗したことはないので特に注意点はない. 

\subsubsection{El Capitanを, MacProのHDDにインストール}
別Macで作業する. 
あらかじめ, MacProから, El Capitanの起動ディスクとして使用するHDDを抜きとっておき, USB-SATA変換ケーブルを介して, 別Macでボリュームとして認識させておく. 
別Macを, 再起動後, Optionキーを押下したまま起動を待つと, 起動ディスク選択画面が表示される. 
ここで, 先ほどブータブルにしたUSBを選択し, インストールを開始する. インストール先として, MacProのHDDを選択して, インストールを進める. 

ここでも, 失敗したことはないので特に注意点はない. ただ, 後述するように, まだよくわかっていないが, もう一度インストールを行う可能性があるので, USBメモリの中身はまだ消さないこと. 
また, インストールが完了したら, 一旦USBメモリを外し, HDDのEl Capitanから別Macが起動できることを, 再起動させて確認しておく. 

\subsubsection{El Capitanのboot.efiを, Pike氏のboot.efiに入れ替え}
別Macで作業する. 
別Macを, 本体のOSで起動し, El CapitanのHDDをボリュームとして認識している状態にする. 
Pike氏のHP\cite{Pike}から, boot.efiをダウンロードする. 
作業の参考に最も適したサイト\cite{Easiest_Reference}の通りに実行できれば成功するが, いつ消えるかもわからないので, 一応書いておくと, 
El Capitanのディレクトリ内の, 
System/Library/CoreServiceに, boot.efiがある. 
これを入れ替えるのだが, ロックされているファイルのため, ロックを外す. 
ターミナルで, 

sudo chflags nouchg (boot.efiのフルパス, ドラッグアンドドロップで行けるのは知らなかった)

を入力する. 
すると, finderの見た目でもロックが外れるので, Pike氏のboot.efiに入れ替える. 
ここで, 参考サイト\cite{Easiest_Reference}では, 

sudo chflags uchg (boot.efiのフルパス, ドラッグアンドドロップで行けるのは知らなかった)

で, 再ロックをかけることになっているが, 現時点で成功したプロセスでは, 再ロックをかけなかった. 
かけるかけないが影響するかは未検証だが, とりあえずうまくいっている状況ではかけていない. 

もう一つ, Macで, 隠しファイルを表示するように設定しておく. 
usr/standalone/i386 の中にも, boot.efiがあり, 同様に置き換える. こちらは最初からロックがかかっていないので, ただドラッグアンドドロップすればいいだけ. 

\subsubsection{MacProにHDDを取り付けて起動}
別Macで, HDDをアンマウントし, MacProにセットし, 起動する. 従来のOSが入ったドライブは取り除いておく. 
ここで, 一度目は, ?マークのフォルダが表示され, OSが認識されなかった. いろいろと原因を調べたが, blessだったり, 権限修復だったり, どれもうまくいかないものばかりで, 
結局, 何もやり方を変えていないにもかかわらず, 同じことを繰り返したところ, うまくいった. どこかで致命的にミスをしただけかもしれないが, 
単純にやり直すという手が今のところ有効. 参考サイト\cite{Easiest_Reference}のコメント欄にも, 同様の症状を訴えている人がいるが, 結局明確な理由はわかっていない. 

ただ, この症状に陥ったHDDから, 別MacをEl Capitan起動しようとしたところ, 正常に起動できなかったので, 何かが失敗している可能性が高い. 

\subsection{後処理}
\subsubsection{スリープモードの設定変更}
MacProは, DDR2メモリに熱問題を抱えており, スリープモードを, ディープスリープに切り替えねばならないらしい\cite{SleepMode_Issue_Doc}. そこで, 
sudo pmset -a hibernatemode X のコマンドを, Xを25に置き換えて実行する. 

\subsubsection{自動アップデートの停止}
ここまでの手法で, El Capitanのインストールに成功したMacProでは, 2018-001のセキュリティアップデートをかけると起動できなくなるらしい\cite{Security_Update_Issue_Doc}. 自動アップデートを停止しておく. 


\begin{thebibliography}{9}

  \bibitem{MacPro_Doc}
  "Mac Pro". 4.1 Mac Pro(1st Generation) .
  \url{https://ja.wikipedia.org/wiki/Mac_Pro#Mac_Pro_(1st_Generation)}

\bibitem{EFI_Issue_Doc}
  Mr. Zaniwoop. "2006/2007 Mac Pro (1,1/2,1) and OS X El Capitan". comment on Jun 8, 2015. Section "Background". 
  \url{https://forums.macrumors.com/threads/2006-2007-mac-pro-1-1-2-1-and-os-x-el-capitan.1890435/}

\bibitem{Summary_Doc}
  Mr. Zaniwoop. "2006/2007 Mac Pro (1,1/2,1) and OS X El Capitan". comment on Jun 8, 2015. Section "How can these Macs run El Capitan?" 
  \url{https://forums.macrumors.com/threads/2006-2007-mac-pro-1-1-2-1-and-os-x-el-capitan.1890435/}


\bibitem{Security_Update_Issue_Doc}
  Mr. Zaniwoop. "2006/2007 Mac Pro (1,1/2,1) and OS X El Capitan". comment on Jun 8, 2015. Section "Security Update 2018-001 for Mac OS X El Capitan" 
  \url{https://forums.macrumors.com/threads/2006-2007-mac-pro-1-1-2-1-and-os-x-el-capitan.1890435/}


\bibitem{Pike}
  "Pike's Universum". Macosxbootloader .
  \url{https://pikeralpha.wordpress.com/macosxbootloader/}

\bibitem{Graphics_Board_Doc}
  "MacPro 2006 純正RadeonHD 5770". 
  \url{https://ameblo.jp/crabman/entry-12110818376.html}

\bibitem{Easiest_Reference}
  "Mac Pro 2006にEl Capitanインストール成功". 
  \url{https://ameblo.jp/crabman/entry-12107497168.html}

\bibitem{SleepMode_Issue_Doc}
  "初代MacProでOS X10.11 El Capitanの落とし穴". 
  \url{http://blog.livedoor.jp/maxcov/archives/65830907.html}

\bibitem{Alternative_GPU}
  "MacPro A1186(2006) – RADEON HD5770". 
  \url{http://blog.pdns.jp/macpro-a11862006-radeon-hd5770/}


\end{thebibliography}
\end{document}
