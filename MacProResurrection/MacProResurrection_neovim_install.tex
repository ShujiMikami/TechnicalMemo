\documentclass{jsarticle}
\usepackage[dvipdfmx]{hyperref}
\usepackage{pxjahyper}
\hypersetup{colorlinks=true,linkcolor=blue}

\title{MacPro1,1(2006) El Capitan に, neovimをインストール}

\begin{document}
\maketitle

\section{背景}
MacPro2006に, El Capitanのインストールが成功した.
一方で, 最近テキスト編集環境を, vimから, neovimに乗換を進めており, MacPro2006に対しても, neovimを入れることにした. 
しかし, El Capitanとはいえ, もはや2020年1月時点では, 4世代前のOSであり, いくつか引っかかる点が出てしまったので, 
ここに記録する. 

\section{概要}
neovimをインストールし, checkhelthがOKになるまでのインストール作業を記録する. 

\section{結論}
ホームページのバイナリインストーラは使えないものの, Homebrewを使用したインストールは可能. 
pythonのSSL認証が失敗するが, アップデートコマンドで対応可能. 

\section{制約}
これと言ってない

\section{残存不明点}
発生した問題が, El Capitan要因なのか, それともハードウェア要因なのか原因がわからない

\subsection{作業概要}
Homebrewでneovimをインストールし, python2, python3をインストール, pythonのSSL認証をアップグレード

\subsection{手順}
\subsubsection{neovimのインストール}
普通は, neovimプロジェクトのホームページから, バイナリをダウンロードすればよい. 
事実, High SierraがインストールされているMacBookAirや, Windows10マシンでは問題なく成功した. 
しかし, MacPro2006では, ダウンロードして解凍したバイナリを実行しようとしても, Illegal Instraction 4というメッセージが出て, 起動できなかったり, ファイルが壊れていて実行できない, などのメッセージで実行できない. 
ネットで見る限り, コンパイラのバージョンが新しいがゆえに, OSが理解できないメッセージが含まれている, とのことで, おそらくEl Capitan依存なのだろうと考えられる. 
つまりは, ソースからビルドすれば動くと思われ, Homebrewでのインストールに切り替えた. 
結果, かなり時間がかかったものの, 無事, 最新リリースである, 0.4.3がインストールされた. 

\subsubsection{pythonのインストール}
全く問題ない. ホームページから最新バイナリインストーラを持ってきてインストールするだけ

\subsubsection{pythonのSSL認証をアップグレード}
ここで, 普通であれば, pipで, pynvimをインストールして完了となるはずであるが, pipでのインストールを行おうとすると, 大量のエラーを吐き, 失敗する. 
エラー内容の中に, SSL: CERTIFICATE\_VERIFY\_FAILEDというメッセージがあり, これで検索してみると, いくつか情報が見つかるが, 
https://qiita.com/TakahiroNozawa/items/89e80201843c681f6e87
がわかりやすく, いう通り, pip install --upgrade certifiを, pip2, pip3それぞれで実行してやれば, 無事, pynvimをインストールできる. 
なんでなのかは未調査


\end{document}
